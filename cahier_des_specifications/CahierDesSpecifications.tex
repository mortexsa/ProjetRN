\documentclass{article}
\usepackage[latin1]{inputenc}    
\usepackage[T1]{fontenc}
\usepackage[french]{babel}
\usepackage{graphicx}
\newcommand\tab[1][1cm]{\hspace*{#1}}
\usepackage{geometry}
\geometry{hmargin=3.45cm,vmargin=3cm}
\usepackage{color}
\definecolor{myblue}{rgb}{0, 0, 1}

\begin{document}

\newpage
\title{Cahier des spécifications\\Sujet 10 : Réseau de Neurones}
\author{Thibaut \bsc{Pepin}\\Soumia \bsc{Rezgui}\\Isaac \bsc{Szulek}\\Severine \bsc{Selaquet}\\Anthony \bsc{Montigne}\\Arezki \bsc{Slimani}}
\maketitle

\newpage

\tableofcontents

\newpage

\section{Structures de Données}
	\subsection{Réseau de neurones}
	Pour le choix de notre structure de données nous avons rapidement adopté la représentation d'un réseau de neurone comme un ensemble de couches et non comme un ensemble de neurone individuel. Chaque Couche devait donc contenir l'ensemble des activation, des biais et des poids des neurones de cette couches sous la forme de vecteurs pour les activations et les biais et sous forme de matrice pour les poids, nous les avons integré dans une structure couche comme tableau a une ou deux dimensions car les calculs a effectuer necessite d'acceder aux éléments sans ordre particulier.
	
	Afin de stocker l'ensemble des couches constituant le réseau de neurone, nous avons opté pour une liste doublement chainé, en effet, les deux algorithme necessitant d'acceder aux contenues des couches vont propager des information de la premiere couche a la derniere couche pour l'algorithme de propagation et de la derniere couches a la premiere couche pour l'algorithme de rétro-propagation, d'ou le double chainage.
	
	Afin de regrouper les informations générales d'un réseau de neurone, nous avons créer une petite structure contenant toutes sortes d'informations comme la date de création.
	
	Un réseau de neurone est donc constitué d'une structure d'information ainsi que d'un pointeur sur la premiere couche et d'un autre pointeur sur la derniere couche.

	\begin{flushleft}
	
		struct COUCHE	//structure représentant une couche (ensemble de neurones) dans un réseau de neurone\\
		\{\\
			\tab \textcolor{myblue}{\textbf{float*}} A;	//tableau représentant le vecteur des activation des neurone d'une couche.Possède autant d'élements que de neurone présent dans la couche\\
			\tab \textcolor{myblue}{\textbf{float*}} B;	//tableau représentant le vecteur des biais des neurone d'une couche. De meme taille que A\\
			\tab \textcolor{myblue}{\textbf{float**}} W;	//tableau a deux dimensions représentant la matrice des poids entre les neurones de la couches précedente et les neurones de cette couche. Ce tableau est donc de taille (taille de la couche actuelle x taille de la couche précedente) et est composé d'élement $w_{ij}$ ou j est le numéro du neurone de la couches précedente et i le numéro du neurone de la couche actuelle.\\
			\tab \textcolor{myblue}{\textbf{int}} taille;	//Indique le nombre de neurones présent dans cette couche\\
			\tab \textcolor{myblue}{\textbf{float}} DELTA;      //tableau représentant le vecteur des modifications a apporter aux biais de cette couches lors de la rétro-propagation des neurone d'une couche.De meme taille que A\\
			\tab \textcolor{myblue}{\textbf{float**}} DELTA\_M; //tableau a deux dimensions représentant la matrice des modifications a apporter aux poids de cette couches lors de la rétro-propagation des neurone d'une couche. De meme taille que W\\
			\medbreak
			\tab \textcolor{myblue}{\textbf{struct COUCHE*}} prec;	//pointeur sur la couche précédente.\\
			\tab \textcolor{myblue}{\textbf{struct COUCHE*}} suiv;	//pointeur sur la couche suivante.\\
		\};\\
		\bigbreak
		typedef struct COUCHE COUCHE;\\
		typedef COUCHE* Liste\_couche;\\
		\bigbreak
		struct INFO\_RN	//structure représentant les différentes informations caractérisant un réseau de neurones.\\
		\{\\
			\tab \textcolor{myblue}{\textbf{char**}} etiquettes;	//tableau de chaine de caractère comprenant la signification des neurones de sortie.\\
			\tab \textcolor{myblue}{\textbf{char*}} nom;	//nom donné au réseau de neurones.\\
			\tab \textcolor{myblue}{\textbf{char*}} date;	//date de création du réseau de neurones.\\
			\tab \textcolor{myblue}{\textbf{int}} reussite;	//nombre de fois ou le réseau de neurone a obtnenue la réponse attendue lors de l'apprentissage.\\
			\tab \textcolor{myblue}{\textbf{int}} echec;	//nombre de fois ou le réseau de neurone n'a pas obtnenue la réponse attendue lors de l'apprentissage.\\
		\};\\
		\bigbreak
		typedef struct INFO\_RN INFO\_RN;\\
		\bigbreak
		struct RN	//structure représentant le réseau de neurones, contenant l'ensemble des couches et les informations du réseau de neurones.\\
		\{\\
			\tab \textcolor{myblue}{\textbf{INFO\_RN}} info;	//les informations du réseau de neurones.\\
			\tab \textcolor{myblue}{\textbf{Liste\_couche}} couche\_deb;	//pointeur sur la 1ere couche du RN.\\
			\tab \textcolor{myblue}{\textbf{Liste\_couche}} couche\_fin;    //pointeur sur la derniere couche du RN.\\
		\};\\
		\bigbreak
		typedef struct RN RN;
		
	\end{flushleft}
	
	\subsection{Image}
	Afin de représenté une image, nous avons opté pour un tableau a une dimension de pixel afin de se rapprocher de la structure des vecteurs d'activations. Chaque pixel étant une structure contenant la quantité de rouge, de vert et de bleu présent dans un nombre entre 0 et 255 d'ou le type char qui convient parfaitement pour des valeur dans cet intervalle.
	\begin{flushleft}
		typedef struct Pixel\\
			\{\\
				\tab \textcolor{myblue}{\textbf{unsigned char}} r,g,b;\\
			\} Pixel;
		\bigbreak
		typedef struct Image\\
			\{\\
				\tab \textcolor{myblue}{\textbf{int}} w,h;\\
				\tab \textcolor{myblue}{\textbf{Pixel*}} dat;\\
			\} Image;
	\end{flushleft}
	
\section{Signatures des Fonctions}
	\subsection{Gestionnaires d'apprentissage}
		\begin{flushleft}
			\textcolor{myblue}{\textbf{void}} BackProp(RN*, \textcolor{myblue}{\textbf{Image*}} ,\textcolor{myblue}{\textbf{char*}}, \textcolor{myblue}{\textbf{float}});\\
			\textcolor{myblue}{\textbf{void}} SigmoidePrimeZ(\textcolor{myblue}{\textbf{float*}} in, \textcolor{myblue}{\textbf{float**}} out, \textcolor{myblue}{\textbf{int}} taille);\\
			\textcolor{myblue}{\textbf{void}} MultiplicationMatricielleTransposeeTM(\textcolor{myblue}{\textbf{float**}},  \textcolor{myblue}{\textbf{float*}},  \textcolor{myblue}{\textbf{float*}},  \textcolor{myblue}{\textbf{int}},  \textcolor{myblue}{\textbf{int}});\\
			\textcolor{myblue}{\textbf{void}} MultiplicationMatricielleTransposeeMT(\textcolor{myblue}{\textbf{float*}},  \textcolor{myblue}{\textbf{float*}},  \textcolor{myblue}{\textbf{float**}},  \textcolor{myblue}{\textbf{int}},  \textcolor{myblue}{\textbf{int}});\\
			\textcolor{myblue}{\textbf{void}} Hadamard(\textcolor{myblue}{\textbf{float**}},  \textcolor{myblue}{\textbf{float*}},  \textcolor{myblue}{\textbf{float*}},  \textcolor{myblue}{\textbf{int}});\\
			\textcolor{myblue}{\textbf{void}} fct\_cout(\textcolor{myblue}{\textbf{RN}}, \textcolor{myblue}{\textbf{char*}});\\
			\textcolor{myblue}{\textbf{void}} ModifPoids(\textcolor{myblue}{\textbf{float**}},  \textcolor{myblue}{\textbf{float**}},  \textcolor{myblue}{\textbf{int}},  \textcolor{myblue}{\textbf{int}},  \textcolor{myblue}{\textbf{int}});\\
			\textcolor{myblue}{\textbf{void}} ModifBiais(\textcolor{myblue}{\textbf{float*}},  \textcolor{myblue}{\textbf{float*}},  \textcolor{myblue}{\textbf{int}},  \textcolor{myblue}{\textbf{int}});
		\end{flushleft}
		
	\subsection{Gestionnaire d'entrées sorties}
		\begin{flushleft}
			\textcolor{myblue}{\textbf{Image*}} ChargerBmp(\textcolor{myblue}{\textbf{const char*}} fichier);\\
			\textcolor{myblue}{\textbf{Image*}} ChargerMnist(\textcolor{myblue}{\textbf{const char*}} fichier);\\
			\textcolor{myblue}{\textbf{int}} Sauver(\textcolor{myblue}{\textbf{Image*}},\textcolor{myblue}{\textbf{const char*}} fichier);\\
			\textcolor{myblue}{\textbf{Image*}} NouvelleImage(\textcolor{myblue}{\textbf{int}} w,\textcolor{myblue}{\textbf{int}} h);\\
			\textcolor{myblue}{\textbf{Image*}} CopieImage(\textcolor{myblue}{\textbf{Image*}});\\
			\textcolor{myblue}{\textbf{void}} SetPixel(\textcolor{myblue}{\textbf{Image*}},\textcolor{myblue}{\textbf{int}} i,\textcolor{myblue}{\textbf{int}} j,Pixel p);\\
			\textcolor{myblue}{\textbf{Pixel}} GetPixel(\textcolor{myblue}{\textbf{Image*}},\textcolor{myblue}{\textbf{int}} i,\textcolor{myblue}{\textbf{int}} j);\\
			\textcolor{myblue}{\textbf{void}} DelImage(Image*);\\
			\textcolor{myblue}{\textbf{char*}} ChargerEtiquette(\textcolor{myblue}{\textbf{const char*}} fichier);\\
			\textcolor{myblue}{\textbf{RN*}} ChargerRN(\textcolor{myblue}{\textbf{const char*}} fichier);\\
			\textcolor{myblue}{\textbf{void}} SaveRN(\textcolor{myblue}{\textbf{RN}});
		\end{flushleft}
	
	\subsection{Gestionnaire de l'apprentissage}
		\begin{flushleft}
			\textcolor{myblue}{\textbf{RN*}} initialisation(\textcolor{myblue}{\textbf{INFO\_RN}});\\
			\textcolor{myblue}{\textbf{void}} AjoutFin(RN, \textcolor{myblue}{\textbf{float*}},  \textcolor{myblue}{\textbf{float**}});\\
			\textcolor{myblue}{\textbf{void}} Traitement(\textcolor{myblue}{\textbf{Image*}}, \textcolor{myblue}{\textbf{RN}});\\
			\textcolor{myblue}{\textbf{char**}} Reconnaissance(\textcolor{myblue}{\textbf{RN}});\\
			\textcolor{myblue}{\textbf{void}} MultiplicationMatricielle(\textcolor{myblue}{\textbf{float**}},  \textcolor{myblue}{\textbf{float**}},  \textcolor{myblue}{\textbf{float**}},  \textcolor{myblue}{\textbf{int}},  \textcolor{myblue}{\textbf{int}},  \textcolor{myblue}{\textbf{int}});\\
			\textcolor{myblue}{\textbf{void}} AdditionVecteurVecteur(\textcolor{myblue}{\textbf{float*}},  \textcolor{myblue}{\textbf{float*}},  \textcolor{myblue}{\textbf{float*}},  \textcolor{myblue}{\textbf{int}});\\
			\textcolor{myblue}{\textbf{void}} SigmoideV(\textcolor{myblue}{\textbf{float*}},  \textcolor{myblue}{\textbf{float*}},  \textcolor{myblue}{\textbf{int}});\\
			\textcolor{myblue}{\textbf{float}} Sigmoide(\textcolor{myblue}{\textbf{float}});
		\end{flushleft}		
		
\end{document}
